\documentclass{beamer}
\usepackage[utf8]{inputenc}

\usetheme{Madrid}
\usecolortheme{default}
\usepackage{amsmath,amssymb,amsfonts,amsthm}
\usepackage{mathtools}
\usepackage{txfonts}
\usepackage{tkz-euclide}
\usepackage{listings}
\usepackage{adjustbox}
\usepackage{array}
\usepackage{gensymb}
\usepackage{tabularx}
\usepackage{gvv}
\usepackage{lmodern}
\usepackage{circuitikz}
\usepackage{tikz}
\lstset{literate={·}{{$\cdot$}}1 {λ}{{$\lambda$}}1 {→}{{$\to$}}1}
\usepackage{graphicx}

\setbeamertemplate{page number in head/foot}[totalframenumber]

\usepackage{tcolorbox}
\tcbuselibrary{minted,breakable,xparse,skins}



\definecolor{bg}{gray}{0.95}
\DeclareTCBListing{mintedbox}{O{}m!O{}}{%
  breakable=true,
  listing engine=minted,
  listing only,
  minted language=#2,
  minted style=default,
  minted options={%
    linenos,
    gobble=0,
    breaklines=true,
    breakafter=,,
    fontsize=\small,
    numbersep=8pt,
    #1},
  boxsep=0pt,
  left skip=0pt,
  right skip=0pt,
  left=25pt,
  right=0pt,
  top=3pt,
  bottom=3pt,
  arc=5pt,
  leftrule=0pt,
  rightrule=0pt,
  bottomrule=2pt,
  toprule=2pt,
  colback=bg,
  colframe=orange!70,
  enhanced,
  overlay={%
    \begin{tcbclipinterior}
    \fill[orange!20!white] (frame.south west) rectangle ([xshift=20pt]frame.north west);
    \end{tcbclipinterior}},
  #3,
}
\lstset{
    language=C,
    basicstyle=\ttfamily\small,
    keywordstyle=\color{blue},
    stringstyle=\color{orange},
    commentstyle=\color{green!60!black},
    numbers=left,
    numberstyle=\tiny\color{gray},
    breaklines=true,
    showstringspaces=false,
}
%------------------------------------------------------------
%This block of code defines the information to appear in the
%Title page
\title %optional
{5.4.22}
\date{29 September, 2025}
%\subtitle{A short story}

\author % (optional)
{INDHIRESH S - EE25BTECH11027}

\begin{document}

\frame{\titlepage}

\begin{frame}{Question}
Using elementary transformations, find the inverse of the following matrices
\begin{align*}
    \myvec{6&-3\\-2&1}
\end{align*}
\end{frame}

\begin{frame}[allowframebreaks] 
\frametitle{Equation}
    \centering
    \label{tab:parameters}
Let the given matrix  be:
\begin{align}
   \Vec{A}=\myvec{6&-3\\-2&1}
\end{align}

\end{frame}

\begin{frame}
Now finding the inverse of a matrix by elementary operation.\\
Now forming the augmented matrix $[\Vec{A}|\Vec{I}]$

\begin{align}
[\Vec{A}|\Vec{I}]=\augvec{2}{2}{6&-3&1&0\\-2&1&0&1}
\end{align}

\begin{align}
   \augvec{2}{2}{6&-3&1&0\\-2&1&0&1}\xleftrightarrow{R_2\longleftarrow R_2+\frac{1}{3}R_1}  \augvec{2}{2}{6&-3&1&0\\0&0&\frac{1}{3}&1}
\end{align}
From above we can observe that the rank of the left-side augmented matrix is 1\\
Therefore the matrix $\Vec{A}$ is singular and hence the inverse does not exist for the given matrix\\
\end{frame}



\begin{frame}[fragile]
    \frametitle{C Code}
    \begin{lstlisting}
#include <stddef.h> // For size_t

/**
 * @brief Performs the operation: dest_row = dest_row + (factor * src_row)
 * This is the core operation for elimination.
 * @param dest_row The row to be modified.
 * @param src_row The row used for the operation.
 * @param factor The multiplication factor.
 * @param size The number of elements in the rows.
 */
void add_scaled_row(double* dest_row, const double* src_row, double factor, size_t size) {
    for (size_t i = 0; i < size; i++) {
        dest_row[i] += factor * src_row[i];
    }
}


    \end{lstlisting}
\end{frame}

\begin{frame}[fragile]
    \frametitle{C Code}
    \begin{lstlisting}
 /**
 * @brief Scales a row by multiplying each element by a factor.
 * Used for normalization (making pivots equal to 1).
 * @param row The row to be scaled.
 * @param factor The scaling factor.
 * @param size The number of elements in the row.
 */
void scale_row(double* row, double factor, size_t size) {
    for (size_t i = 0; i < size; i++) {
        row[i] *= factor;
    }
}
    \end{lstlisting}
\end{frame}



\begin{frame}[fragile]
    \frametitle{Python Code}
    \begin{lstlisting}
import numpy as np
import ctypes

# Define a pointer type for a double array
DOUBLE_PTR = ctypes.POINTER(ctypes.c_double)

c_lib = ctypes.CDLL('./inverse.so')

# Define argtypes for add_scaled_row(double*, double*, double, size_t)
c_lib.add_scaled_row.argtypes = [DOUBLE_PTR, DOUBLE_PTR, ctypes.c_double, ctypes.c_size_t]
c_lib.add_scaled_row.restype = None

# Define argtypes for scale_row(double*, double, size_t)
c_lib.scale_row.argtypes = [DOUBLE_PTR, ctypes.c_double, ctypes.c_size_t]
c_lib.scale_row.restype = None



    \end{lstlisting}
\end{frame}

\begin{frame}[fragile]
    \frametitle{Python Code}
    \begin{lstlisting}
def invert_matrix_with_c(A):
    """
    Inverts a 2x2 matrix using Python logic and C row operations.
    """
    if A.shape != (2, 2):
        return "Input must be a 2x2 matrix.", False
    
    # 1. Create the 2x4 augmented matrix [A|I]
    I = np.identity(2)
    aug = np.concatenate((A, I), axis=1).astype(np.float64)
    num_cols = aug.shape[1]

    # Get C-compatible pointers to the start of each row's data
    row0_ptr = aug[0].ctypes.data_as(DOUBLE_PTR)
    row1_ptr = aug[1].ctypes.data_as(DOUBLE_PTR)

    # --- Python Logic controlling C Functions ---
    
    
    \end{lstlisting}
\end{frame}

\begin{frame}[fragile]
    \frametitle{Python Code}
    \begin{lstlisting}
# 2. Forward Elimination
    # Python calculates the factor
    factor = -aug[1, 0] / aug[0, 0]
    # C performs the operation: R2 -> R2 + factor * R1
    c_lib.add_scaled_row(row1_ptr, row0_ptr, factor, num_cols)
    
    # 3. Check for Singularity (in Python)
    if abs(aug[1, 1]) < 1e-9:
        return "Matrix is singular; inverse does not exist.", False

    # 4. Backward Elimination
    # Python calculates the factor
    factor = -aug[0, 1] / aug[1, 1]
    # C performs the operation: R1 -> R1 + factor * R2
    c_lib.add_scaled_row(row0_ptr, row1_ptr, factor, num_cols)

    
    \end{lstlisting}
\end{frame}

\begin{frame}[fragile]
    \frametitle{Python Code}
    \begin{lstlisting}
# 5. Normalization
    # Python calculates the scaling factor for row 0
    scale_factor_r0 = 1.0 / aug[0, 0]
    # C scales the row: R1 -> R1 / aug[0, 0]
    c_lib.scale_row(row0_ptr, scale_factor_r0, num_cols)
    
    # Python calculates the scaling factor for row 1
    scale_factor_r1 = 1.0 / aug[1, 1]
    # C scales the row: R2 -> R2 / aug[1, 1]
    c_lib.scale_row(row1_ptr, scale_factor_r1, num_cols)
    
    # 6. Extract the inverse
    inverse = aug[:, 2:]
    return inverse, True



    \end{lstlisting}
\end{frame}

\begin{frame}[fragile]
    \frametitle{Python Code}
    \begin{lstlisting}
# --- Main execution ---
if __name__ == "__main__":
    matrix = np.zeros((2, 2))
    print("Enter the elements of the 2x2 matrix:")
    for i in range(2):
        for j in range(2):
            value = float(input(f"Enter element [{i}][{j}]: "))
            matrix[i, j] = value

    print("\nInput Matrix:\n", matrix)
    print("--------------------------------")
    
    result, success = invert_matrix_with_c(matrix)
    
    if success:
        print("Inverse Matrix Found:\n", result)
    else:
        print("Result:", result)
    \end{lstlisting}
\end{frame}

\end{document}