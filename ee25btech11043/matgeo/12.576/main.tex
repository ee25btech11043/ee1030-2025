\documentclass[journal]{IEEEtran}
\usepackage[a5paper, margin=10mm, onecolumn]{geometry}
\usepackage{lmodern}
\usepackage{tfrupee}
\setlength{\headheight}{1cm}
\setlength{\headsep}{0mm}

\usepackage{gvv-book}
\usepackage{gvv}
\usepackage{cite}
\usepackage{amsmath,amssymb,amsfonts,amsthm}
\usepackage{algorithmic}
\usepackage{graphicx}
\usepackage{textcomp}
\usepackage{xcolor}
\usepackage{txfonts}
\usepackage{listings}
\usepackage{enumitem}
\usepackage{mathtools}
\usepackage{gensymb}
\usepackage{comment}
\usepackage[breaklinks=true]{hyperref}
\usepackage{tkz-euclide}
\usepackage{listings}
\def\inputGnumericTable{}
\usepackage[latin1]{inputenc}
\usepackage{color}
\usepackage{array}
\usepackage{longtable}
\usepackage{calc}
\usepackage{multirow}
\usepackage{hhline}
\usepackage{ifthen}
\usepackage{lscape}
\usepackage{xparse}

\bibliographystyle{IEEEtran}

\title{12.576}
\author{EE25BTECH11043 - Nishid Khandagre}

\begin{document}
\maketitle

\renewcommand{\thefigure}{\theenumi}
\renewcommand{\thetable}{\theenumi}

\numberwithin{equation}{enumi}
\numberwithin{figure}{enumi}

\textbf{Question}:\
If the characteristic polynomial and minimal polynomial of a square matrix $\vec{A}$ are $(\lambda-1)(\lambda+1)^4(\lambda-2)^5$ and $(\lambda-1)(\lambda+1)(\lambda-2)$, respectively, then the rank of the matrix $\vec{A}+\vec{I}$ is?

\textbf{Solution: }
Given:
\begin{align}
\chi_A(\lambda) &= (\lambda-1)(\lambda+1)^4(\lambda-2)^5 \\
m_A(\lambda) &= (\lambda-1)(\lambda+1)(\lambda-2)
\end{align}

Size of $\vec{A}$=degree of $\chi_A$
\begin{align}
\deg\chi_A &= 1+4+5 = 10
\end{align}
Thus, $\vec{A}$ is a $10 \times 10$ matrix.\\\\
The minimal polynomial $m_A(\lambda)$ has simple roots \\(all linear factors with exponent 1).
\begin{align}
m_A(\lambda) = (\lambda-1)(\lambda+1)(\lambda-2)
\end{align}
Since all roots are distinct, the matrix $\vec{A}$ is diagonalizable.\\\\

Eigenvalues of $\vec{A}+\vec{I}$ and the zero-eigenspace:\\

If $\lambda$ is an eigenvalue of $\vec{A}$, then $\lambda+1$ is an eigenvalue of .\\\\
The eigenvalue 0 of $\vec{A}+\vec{I}$ corresponds to the eigenvalue $-1$ of $\vec{A}$.
From $\chi_A(\lambda)$, the algebraic multiplicity of $\lambda=-1$ is 4.\\
Since $\vec{A}$ is diagonalizable, the geometric multiplicity of $\lambda=-1$ is equal to its algebraic multiplicity, which is 4.\\
Therefore, the geometric multiplicity of 0 for $\vec{A}+\vec{I}$ is 4.
\begin{align}
\operatorname{nullity}(\vec{A}+\vec{I}) &= \dim\ker(\vec{A}+\vec{I}) = 4
\end{align}

Rank-nullity theorem:
\begin{align}
\operatorname{rank}(\vec{A}+\vec{I}) + \operatorname{nullity}(\vec{A}+\vec{I}) &= n
\end{align}
Here, $n=10$ and $\operatorname{nullity}(\vec{A}+\vec{I})=4$.
\begin{align}
\operatorname{rank}(\vec{A}+\vec{I}) &= 10 - 4 \\
&= 6
\end{align}

Thus, the rank of the matrix $\vec{A}+\vec{I}$ is 6.

\end{document}
