\documentclass{beamer}
\usepackage[utf8]{inputenc}

\usetheme{Madrid}
\usecolortheme{default}
\usepackage{amsmath,amssymb,amsfonts,amsthm}
\usepackage{txfonts}
\usepackage{tkz-euclide}
\usepackage{listings}
\usepackage{adjustbox}
\usepackage{array}
\usepackage{tabularx}
\usepackage{gvv}
\usepackage{lmodern}
\usepackage{circuitikz}
\usepackage{tikz}
\usepackage{graphicx}

\setbeamertemplate{page number in head/foot}[totalframenumber]

\usepackage{tcolorbox}
\tcbuselibrary{minted,breakable,xparse,skins}

\definecolor{bg}{gray}{0.95}
\DeclareTCBListing{mintedbox}{O{}m!O{}}{%
breakable=true,
listing engine=minted,
listing only,
minted language=#2,
minted style=default,
minted options={%
linenos,
gobble=0,
breaklines=true,
breakafter=,,
fontsize=\small,
numbersep=8pt,
#1},
boxsep=0pt,
left skip=0pt,
right skip=0pt,
left=25pt,
right=0pt,
top=3pt,
bottom=3pt,
arc=5pt,
leftrule=0pt,
rightrule=0pt,
bottomrule=2pt,
toprule=2pt,
colback=bg,
colframe=orange!70,
enhanced,
overlay={%
\begin{tcbclipinterior}
\fill[orange!20!white] (frame.south west) rectangle ([xshift=20pt]frame.north west);
\end{tcbclipinterior}},
#3,
}
\lstset{
language=C,
basicstyle=\ttfamily\small,
keywordstyle=\color{blue},
stringstyle=\color{orange},
commentstyle=\color{green!60!black},
numbers=left,
numberstyle=\tiny\color{gray},
breaklines=true,
showstringspaces=false,
}

\title
{12.576}
\date{October 10, 2025}
\author
{EE25BTECH11043 - Nishid Khandagre}

\begin{document}

\frame{\titlepage}

\begin{frame}{Question}
If the characteristic polynomial and minimal polynomial of a square matrix $\vec{A}$ are $(\lambda-1)(\lambda+1)^4(\lambda-2)^5$ and $(\lambda-1)(\lambda+1)(\lambda-2)$, respectively, then the rank of the matrix $\vec{A}+\vec{I}$ is?
\end{frame}

\begin{frame}{Solution}
Given:
\begin{align}
\chi_A(\lambda) &= (\lambda-1)(\lambda+1)^4(\lambda-2)^5 \\
m_A(\lambda) &= (\lambda-1)(\lambda+1)(\lambda-2)
\end{align}

Size of $\vec{A}$=degree of $\chi_A$
\begin{align}
\deg\chi_A &= 1+4+5 = 10
\end{align}
Thus, $\vec{A}$ is a $10 \times 10$ matrix.
\end{frame}

\begin{frame}{Solution}
The minimal polynomial $m_A(\lambda)$ has simple roots (all linear factors with exponent 1).
\begin{align}
m_A(\lambda) = (\lambda-1)(\lambda+1)(\lambda-2)
\end{align}
Since all roots are distinct, the matrix $\vec{A}$ is diagonalizable.
\end{frame}

\begin{frame}{Solution}
Eigenvalues of $\vec{A}+\vec{I}$ and the zero-eigenspace:\\

If $\lambda$ is an eigenvalue of $\vec{A}$, then $\lambda+1$ is an eigenvalue of $\vec{A}+\vec{I}$.\\

The eigenvalue 0 of $\vec{A}+\vec{I}$ corresponds to the eigenvalue $-1$ of $\vec{A}$.

From $\chi_A(\lambda)$, the algebraic multiplicity of $\lambda=-1$ is 4.\\

Since $\vec{A}$ is diagonalizable, the geometric multiplicity of $\lambda=-1$ is equal to its algebraic multiplicity, which is 4.\\

Therefore, the geometric multiplicity of 0 for $\vec{A}+\vec{I}$ is 4.
\begin{align}
\operatorname{nullity}(\vec{A}+\vec{I}) &= \dim\ker(\vec{A}+\vec{I}) = 4
\end{align}
\end{frame}

\begin{frame}{Solution}
Rank-nullity theorem:
\begin{align}
\operatorname{rank}(\vec{A}+\vec{I}) + \operatorname{nullity}(\vec{A}+\vec{I}) &= n
\end{align}
Here, $n=10$ and $\operatorname{nullity}(\vec{A}+\vec{I})=4$.
\begin{align}
\operatorname{rank}(\vec{A}+\vec{I}) &= 10 - 4 \\
&= 6
\end{align}
Thus, the rank of the matrix $\vec{A}+\vec{I}$ is 6.
\end{frame}

\end{document}