\documentclass[journal]{IEEEtran}
\usepackage[a5paper, margin=10mm, onecolumn]{geometry}
\usepackage{lmodern}
\usepackage{tfrupee}
\setlength{\headheight}{1cm}
\setlength{\headsep}{0mm}

\usepackage{gvv-book}
\usepackage{gvv}
\usepackage{cite}
\usepackage{amsmath,amssymb,amsfonts,amsthm}
\usepackage{algorithmic}
\usepackage{graphicx}
\usepackage{textcomp}
\usepackage{xcolor}
\usepackage{txfonts}
\usepackage{listings}
\usepackage{enumitem}
\usepackage{mathtools}
\usepackage{gensymb}
\usepackage{comment}
\usepackage[breaklinks=true]{hyperref}
\usepackage{tkz-euclide}
\usepackage{listings}
\def\inputGnumericTable{}
\usepackage[latin1]{inputenc}
\usepackage{color}
\usepackage{array}
\usepackage{longtable}
\usepackage{calc}
\usepackage{multirow}
\usepackage{hhline}
\usepackage{ifthen}
\usepackage{lscape}
\usepackage{xparse}

\bibliographystyle{IEEEtran}

\title{12.784}
\author{EE25BTECH11043 - Nishid Khandagre}

\begin{document}
\maketitle

\renewcommand{\thefigure}{\theenumi}
\renewcommand{\thetable}{\theenumi}

\numberwithin{equation}{enumi}
\numberwithin{figure}{enumi}

\textbf{Question}:
Let A = $(a_{ij})$ be a 3 $\times$ 3 real matrix such that
A $\myvec{1\\2\\1}$ = 2$\myvec{1\\2\\1}$, A $\myvec{0\\1\\1}$ = 2$\myvec{0\\1\\1}$ and A$\myvec{-1\\1\\0}$ = 4$\myvec{-1\\1\\0}$.
If $m$ is the degree of the minimal polynomial of A, then $a_{11} + a_{21} + a_{31} + m$ equals

\textbf{Solution: }
Given eigen relations:
\begin{align}
\vec{A} \myvec{1\\2\\1} = 2\myvec{1\\2\\1} \\
\vec{A} \myvec{0\\1\\1} = 2\myvec{0\\1\\1} \\
\vec{A} \myvec{-1\\1\\0} = 4\myvec{-1\\1\\0}
\end{align}
The eigenvectors are :
\begin{align}
\vec{v_1} = \myvec{1\\2\\1}, \quad
\vec{v_2} = \myvec{0\\1\\1}, \quad
\vec{v_3} = \myvec{-1\\1\\0}
\end{align}
as $\vec{v_1},\vec{v_2},\vec{v_3}$ are linearly independent

Form the matrix $\vec{P}$ with these eigenvectors as columns:
\begin{align}
\vec{P} = \myvec{
1 & 0 & -1 \\
2 & 1 & 1 \\
1 & 1 & 0
}
\end{align}
The inverse of $\vec{P}$ is:
\begin{align}
\vec{P}^{-1} = 
\myvec{
\frac{1}{2} & \frac{1}{2} & -\frac{1}{2} \\
-\frac{1}{2} & -\frac{1}{2} & \frac{3}{2} \\
-\frac{1}{2} & \frac{1}{2} & -\frac{1}{2}
}
\end{align}

\begin{align}
\vec{P}=\myvec{\vec{v_1}&\vec{v_2}&\vec{v_3}}\\
\vec{A}\vec{P}=\myvec{\vec{A}\vec{v_1}&\vec{A}\vec{v_2}&\vec{A}\vec{v_3}}\\
\vec{A}\vec{P}=\myvec{2\vec{v_1}&2\vec{v_2}&4\vec{v_3}}\\
\vec{A}\vec{P}=\vec{P}\vec{D}
\end{align}
\begin{align}
\vec{D}=
\myvec{
2 & 0 & 0 \\
0 & 2 & 0 \\
0 & 0 & 4
}
\end{align}
\begin{align}
\vec{PD} &= \myvec{
1 & 0 & -1 \\
2 & 1 & 1 \\
1 & 1 & 0
}
\begin{pmatrix}
2 & 0 & 0 \\
0 & 2 & 0 \\
0 & 0 & 4
\end{pmatrix} \\
&= \begin{pmatrix}
2 & 0 & -4 \\
4 & 2 & 4 \\
2 & 2 & 0
\end{pmatrix}
\end{align}
Now, compute $\vec{A} = \vec{PD}\vec{P}^{-1}$:
\begin{align}
\vec{A} &= \myvec{
2 & 0 & -4 \\
4 & 2 & 4 \\
2 & 2 & 0
}
\myvec{
\frac{1}{2} & \frac{1}{2} & -\frac{1}{2} \\
-\frac{1}{2} & -\frac{1}{2} & \frac{3}{2} \\
-\frac{1}{2} & \frac{1}{2} & -\frac{1}{2}
}
\end{align}
\begin{align}
\vec{A} = \myvec{
3 & -1 & 1 \\
-1 & 3 & -1 \\
0 & 0 & 2
}
\end{align}
The sum of the first column elements is:
\begin{align}
a_{11} + a_{21} + a_{31} = 3 + (-1) + 0 = 2
\end{align}
The eigenvalues of $\vec{A}$ are 2 (with algebraic multiplicity 2) and 4 (with algebraic multiplicity 1).\\
Since there are three linearly independent eigenvectors, the matrix $\vec{A}$ is diagonalizable. The minimal polynomial is the product of distinct linear factors corresponding to the eigenvalues.
\begin{align}
m_A(x) = (x-2)(x-4)
\end{align}
The degree of the minimal polynomial, $m$ is 2.

\begin{align}
a_{11} + a_{21} + a_{31} + m = 2 + 2 = 4
\end{align}

\end{document}
