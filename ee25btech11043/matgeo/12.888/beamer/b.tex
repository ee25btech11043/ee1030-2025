\documentclass{beamer}
\usepackage[utf8]{inputenc}

\usetheme{Madrid}
\usecolortheme{default}
\usepackage{amsmath,amssymb,amsfonts,amsthm}
\usepackage{txfonts}
\usepackage{tkz-euclide}
\usepackage{listings}
\usepackage{adjustbox}
\usepackage{array}
\usepackage{tabularx}
\usepackage{gvv}
\usepackage{lmodern}
\usepackage{circuitikz}
\usepackage{tikz}
\usepackage{graphicx}

\setbeamertemplate{page number in head/foot}[totalframenumber]

\usepackage{tcolorbox}
\tcbuselibrary{minted,breakable,xparse,skins}

\definecolor{bg}{gray}{0.95}
\DeclareTCBListing{mintedbox}{O{}m!O{}}{%
breakable=true,
listing engine=minted,
listing only,
minted language=#2,
minted style=default,
minted options={%
linenos,
gobble=0,
breaklines=true,
breakafter=,,
fontsize=\small,
numbersep=8pt,
#1},
boxsep=0pt,
left skip=0pt,
right skip=0pt,
left=25pt,
right=0pt,
top=3pt,
bottom=3pt,
arc=5pt,
leftrule=0pt,
rightrule=0pt,
bottomrule=2pt,
toprule=2pt,
colback=bg,
colframe=orange!70,
enhanced,
overlay={%
\begin{tcbclipinterior}
\fill[orange!20!white] (frame.south west) rectangle ([xshift=20pt]frame.north west);
\end{tcbclipinterior}},
#3,
}
\lstset{
language=C,
basicstyle=\ttfamily\small,
keywordstyle=\color{blue},
stringstyle=\color{orange},
commentstyle=\color{green!60!black},
numbers=left,
numberstyle=\tiny\color{gray},
breaklines=true,
showstringspaces=false,
}

\title
{12.888}
\date{October 10, 2025}
\author
{EE25BTECH11043 - Nishid Khandagre}

\begin{document}

\frame{\titlepage}

\begin{frame}{Question}
Which one of the following matrices has eigenvalues 1 and 6?

\begin{itemize}
\item a)$\myvec{5&-2\\-2&2}$
\item b)$\myvec{3&-1\\-2&2}$
\item c)$\myvec{3&-1\\-1&2}$
\item d)$\myvec{2&-1\\-1&3}$
\end{itemize}
\end{frame}

\begin{frame}{Theoretical Solution}
To find the eigenvalues $\lambda$ of a matrix $\vec{M}$, we solve the characteristic equation 
\begin{align}
\det\myvec{\vec{M} - \lambda \vec{I}} = 0
\end{align}

For $\vec{A}=\myvec{5 & -2 \\ -2 & 2}$
\begin{align}
\det\myvec{\vec{A} - \lambda \vec{I}} &= \mydet{
5-\lambda & -2\\
-2 & 2-\lambda
} \\
&= (5-\lambda)(2-\lambda)-(-2)(-2)
\end{align}
\begin{align}
&= \lambda^2 - 7\lambda + 6
\end{align}
\end{frame}

\begin{frame}{Theoretical Solution}
Now, we solve the characteristic equation:
\begin{align}
\lambda^2 - 7\lambda + 6 &= 0 \\
(\lambda - 1)(\lambda - 6) &= 0 \\
\lambda &= 1, 6
\end{align}
The eigenvalues are 1 and 6. This matches the requirement.

For $\vec{B}=\myvec{3 & -1 \\ -2 & 2}$
\begin{align}
\det\myvec{\vec{B} - \lambda \vec{I}} &= \mydet{
3-\lambda & -1\\
-2 & 2-\lambda
} \\
&= (3-\lambda)(2-\lambda)-(-1)(-2) \\
&= \lambda^2 - 5\lambda + 4
\end{align}
\end{frame}

\begin{frame}{Theoretical Solution}
\begin{align}
\lambda^2 - 5\lambda + 4 &= 0 \\
(\lambda - 1)(\lambda - 4) &= 0 \\
\lambda &= 1, 4
\end{align}
The eigenvalues are 1 and 4. This does not match the requirement.

For $\vec{C}=\myvec{3 & -1 \\ -1 & 2}$
\begin{align}
\det\myvec{\vec{C} - \lambda \vec{I}} &= \mydet{
3-\lambda & -1\\
-1 & 2-\lambda
} \\
&= (3-\lambda)(2-\lambda)-(-1)(-1) \\
&= \lambda^2 - 5\lambda + 5
\end{align}
\end{frame}

\begin{frame}{Theoretical Solution}
\begin{align}
\lambda^2 - 5\lambda + 5 &= 0 \\
\lambda &=\frac{5 \pm \sqrt{5}}{2}
\end{align}
The eigenvalues are $\frac{5 + \sqrt{5}}{2}$ and $\frac{5 - \sqrt{5}}{2}$. This does not match the requirement.

For $\vec{D}=\myvec{2 & -1 \\ -1 & 3}$
\begin{align}
\det\myvec{\vec{D} - \lambda \vec{I}} &= \mydet{
2-\lambda & -1\\
-1 & 3-\lambda
} \\
&= (2-\lambda)(3-\lambda)-(-1)(-1) \\
&= \lambda^2 - 5\lambda + 5
\end{align}
\end{frame}

\begin{frame}{Theoretical Solution}
\begin{align}
\lambda^2 - 5\lambda + 5 &= 0 \\
\lambda&= \frac{5 \pm \sqrt{5}}{2}
\end{align}
The eigenvalues are $\frac{5 + \sqrt{5}}{2}$ and $\frac{5 - \sqrt{5}}{2}$. This does not match the requirement.
\\\\
Only matrix $\vec{A}$ has eigenvalues 1 and 6.
\end{frame}

\begin{frame}[fragile]\frametitle{C Code}
\begin{lstlisting}[language=C]
#include <math.h>
#include <stdio.h>

// Function to find eigenvalues of a 2x2 matrix
// mat = [[a, b], [c, d]]
// Returns 0 if real eigenvalues are found, -1 if complex
int findEigenvalues(double a, double b, double c, double d, double *lambda1, double *lambda2) {
    double trace = a + d;
    double det = a * d - b * c;
\end{lstlisting}
\end{frame}

\begin{frame}[fragile]\frametitle{C Code}
\begin{lstlisting}[language=C]
    // Characteristic equation: lambda^2 - (trace)lambda + (det) = 0
    // Using quadratic formula: lambda = (-B ± sqrt(B^2 - 4AC)) / 2A
    // Here A=1, B=-trace, C=det
    double discriminant = trace * trace - 4 * det;

    if (discriminant >= 0) {
        *lambda1 = (trace + sqrt(discriminant)) / 2.0;
        *lambda2 = (trace - sqrt(discriminant)) / 2.0;
        return 0; // Real eigenvalues
    } else {
        // Complex eigenvalues (not relevant for this problem, but good to handle)
        return -1; // Complex eigenvalues
    }
}
\end{lstlisting}
\end{frame}

\begin{frame}[fragile]\frametitle{Python Code using C Shared Output}
\begin{lstlisting}[language=Python]
import ctypes
import numpy as np

# Load the shared library
lib_eigen = ctypes.CDLL("./code27.so")

# Define the argument types and return type for the C function
lib_eigen.findEigenvalues.argtypes = [
    ctypes.c_double, # a
    ctypes.c_double, # b
    ctypes.c_double, # c
    ctypes.c_double, # d
    ctypes.POINTER(ctypes.c_double), # lambda1
    ctypes.POINTER(ctypes.c_double)  # lambda2
]
lib_eigen.findEigenvalues.restype = ctypes.c_int 
\end{lstlisting}
\end{frame}

\begin{frame}[fragile]\frametitle{Python Code using C Shared Output}
\begin{lstlisting}[language=Python]
# Given target eigenvalues
target_eigenvalues = {1.0, 6.0}

matrices = {
    "a": np.array([[5, -2], [-2, 2]], dtype=float),
    "b": np.array([[3, -1], [-2, 2]], dtype=float),
    "c": np.array([[3, -1], [-1, 2]], dtype=float),
    "d": np.array([[2, -1], [-1, 3]], dtype=float)
}

print("Checking matrices for eigenvalues 1 and 6:\n")
for label, mat in matrices.items():
    a, b = mat[0, 0], mat[0, 1]
    c, d = mat[1, 0], mat[1, 1]

    lambda1_result = ctypes.c_double()
    lambda2_result = ctypes.c_double()
\end{lstlisting}
\end{frame}

\begin{frame}[fragile]\frametitle{Python Code using C Shared Output}
\begin{lstlisting}[language=Python]
    # Call the C function
    status = lib_eigen.findEigenvalues(
        a, b, c, d,
        ctypes.byref(lambda1_result),
        ctypes.byref(lambda2_result)
    )
    if status == 0:
        found_eigenvalues = {round(lambda1_result.value, 6), round(lambda2_result.value, 6)}
        print(f"Matrix {label}:\n{mat}")
        print(f"  Calculated eigenvalues: {found_eigenvalues}")
        if found_eigenvalues == target_eigenvalues:
            print("  This matrix has eigenvalues 1 and 6!\n")
        else:
            print("  Eigenvalues do not match 1 and 6.\n")
    else:
        print(f"Matrix {label}:\n{mat}")
        print("Could not find real eigenvalues")
\end{lstlisting}
\end{frame}

\begin{frame}[fragile]\frametitle{Python Code: Direct}
\begin{lstlisting}[language=Python]
import numpy as np

def check_eigenvalues(matrix, target_eigenvalues):
    """
    Calculates the eigenvalues of a 2x2 matrix and checks if they match
    the target eigenvalues.
    """
    eigenvalues, _ = np.linalg.eig(matrix)

    calculated_eigenvalues = set(round(val.real, 6) for val in eigenvalues)

    return calculated_eigenvalues == target_eigenvalues, calculated_eigenvalues
\end{lstlisting}
\end{frame}

\begin{frame}[fragile]\frametitle{Python Code: Direct}
\begin{lstlisting}[language=Python]
# Define the target eigenvalues
target_eigenvalues = {1.0, 6.0}

# Define the matrices as NumPy arrays
matrices = {
    "a": np.array([[5, -2], [-2, 2]]),
    "b": np.array([[3, -1], [-2, 2]]),
    "c": np.array([[3, -1], [-1, 2]]),
    "d": np.array([[2, -1], [-1, 3]])
}

print(f"Searching for a matrix with eigenvalues: {target_eigenvalues}\n")
\end{lstlisting}
\end{frame}

\begin{frame}[fragile]\frametitle{Python Code: Direct}
\begin{lstlisting}[language=Python]
# Iterate through each matrix and check its eigenvalues
for label, matrix in matrices.items():
    print(f"Checking Matrix {label}:\n{matrix}")
    match, calculated_eigs = check_eigenvalues(matrix, target_eigenvalues)

    if match:
        print(f"  Result: This matrix has the target eigenvalues {calculated_eigs}!\n")
    else:
        print(f"  Result: Calculated eigenvalues are {calculated_eigs}, which do NOT match the target.\n")
\end{lstlisting}
\end{frame}

\end{document}