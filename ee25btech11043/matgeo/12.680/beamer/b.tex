\documentclass{beamer}
\usepackage[utf8]{inputenc}
\usetheme{Madrid}
\usecolortheme{default}
\usepackage{amsmath,amssymb,amsfonts,amsthm}
\usepackage{txfonts}
\usepackage{tkz-euclide}
\usepackage{listings}
\usepackage{adjustbox}
\usepackage{array}
\usepackage{tabularx}
\usepackage{gvv}
\usepackage{lmodern}
\usepackage{circuitikz}
\usepackage{tikz}
\usepackage{graphicx}
\setbeamertemplate{page number in head/foot}[totalframenumber]
\usepackage{tcolorbox}
\tcbuselibrary{minted,breakable,xparse,skins}
\definecolor{bg}{gray}{0.95}
\DeclareTCBListing{mintedbox}{O{}m!O{}}{%
breakable=true,
listing engine=minted,
listing only,
minted language=#2,
minted style=default,
minted options={%
linenos,
gobble=0,
breaklines=true,
breakafter=,,
fontsize=\small,
numbersep=8pt,
#1},
boxsep=0pt,
left skip=0pt,
right skip=0pt,
left=25pt,
right=0pt,
top=3pt,
bottom=3pt,
arc=5pt,
leftrule=0pt,
rightrule=0pt,
bottomrule=2pt,
toprule=2pt,
colback=bg,
colframe=orange!70,
enhanced,
overlay={%
\begin{tcbclipinterior}
\fill[orange!20!white] (frame.south west) rectangle ([xshift=20pt]frame.north west);
\end{tcbclipinterior}},
#3,
}
\lstset{
language=C,
basicstyle=\ttfamily\small,
keywordstyle=\color{blue},
stringstyle=\color{orange},
commentstyle=\color{green!60!black},
numbers=left,
numberstyle=\tiny\color{gray},
breaklines=true,
showstringspaces=false,
}

\title
{12.680}
\date{October 10, 2025}
\author
{EE25BTECH11043 - Nishid Khandagre}

\begin{document}

\frame{\titlepage}

\begin{frame}{Question}
The rank of matrix is ?
\begin{align*}
\myvec{
1 & 2 & 2 & 3\\
3 & 4 & 2 & 5\\
5 & 6 & 2 & 7\\
7 & 8 & 2 & 9
}
\end{align*}
\end{frame}

\begin{frame}{Theoretical Solution}
Let the given matrix be $\vec{A}$:
\begin{align}
\vec{A}=\myvec{
1 & 2 & 2 & 3\\
3 & 4 & 2 & 5\\
5 & 6 & 2 & 7\\
7 & 8 & 2 & 9
}
\end{align}

$R_2 \rightarrow R_2-3R_1$ , $R_3 \rightarrow R_3-5R_1$ , $R_4 \rightarrow R_4-7R_1$
    \begin{align}
    \myvec{
    1 & 2 & 2 & 3\\
    0 & -2 & -4 & -4\\
    0 & -4 & -8 & -8\\
    0 & -6 & -12 & -12
    }
    \end{align}
\end{frame}

\begin{frame}{Theoretical Solution}
$R_2 \rightarrow -\frac{1}{2}R_2$ 
    \begin{align}
    \myvec{
    1 & 2 & 2 & 3\\
    0 & 1 & 2 & 2\\
    0 & -4 & -8 & -8\\
    0 & -6 & -12 & -12
    }
    \end{align}

$R_3 \rightarrow R_3+4R_2$ , $R_4 \rightarrow R_4+6R_2$ , $R_1 \rightarrow R_1-2R_2$
    \begin{align}
    \myvec{
    1 & 0 & -2 & -1\\
    0 & 1 & 2 & 2\\
    0 & 0 & 0 & 0\\
    0 & 0 & 0 & 0
    }
    \end{align}
The number of non-zero rows (pivot rows) in the row-echelon form is 2.
Therefore, the rank of the matrix $A$ is 2.
\end{frame}

\begin{frame}[fragile]
\frametitle{C Code}
\begin{lstlisting}[language=C]
#include <stdio.h>
#include <stdlib.h>
#include <math.h>

// Function to swap two rows in a matrix
void swapRows(int *matrix, int r1, int r2, int cols) {
    for (int j = 0; j < cols; j++) {
        int temp = *(matrix + r1 * cols + j);
        *(matrix + r1 * cols + j) = *(matrix + r2 * cols + j);
        *(matrix + r2 * cols + j) = temp;
    }
}
\end{lstlisting}
\end{frame}

\begin{frame}[fragile]
\frametitle{C Code}
\begin{lstlisting}[language=C]
// Function to calculate the rank of a matrix
int calculate_rank(int *matrix, int rows, int cols) {
    int rank = 0;
    int lead = 0; // Current column to process

    for (int r = 0; r < rows && lead < cols; r++) {
        int i = r;
//Find a row with a non-zero element in the current column 'lead'
    while (i < rows && *(matrix + i * cols + lead) == 0) {
        i++;
    }
    if (i == rows) {
        // No pivot found in this column, move to the next column
        lead++;
        r--; // Re-process the current row with the new lead column
        continue;
        }
\end{lstlisting}
\end{frame}

\begin{frame}[fragile]
\frametitle{C Code}
\begin{lstlisting}[language=C]
        // Swap the current row with the pivot row
        swapRows(matrix, r, i, cols);
        // Eliminate other rows
        for (i = 0; i < rows; i++) {
            if (i != r) {
                int factor = *(matrix + i * cols + lead);
                int pivot_val = *(matrix + r * cols + lead);

                if (pivot_val == 0) continue;
                for (int j = lead; j < cols; j++) {
                    *(matrix + i * cols + j) = (pivot_val * *(matrix + i * cols + j)) - (factor * *(matrix + r * cols + j));
                }
            }
        }
        lead++;
    }
\end{lstlisting}
\end{frame}

\begin{frame}[fragile]
\frametitle{C Code}
\begin{lstlisting}[language=C]
    // Count non-zero rows (each non-zero row indicates a pivot)
    for (int i = 0; i < rows; i++) {
        for (int j = 0; j < cols; j++) {
            if (*(matrix + i * cols + j) != 0) {
                rank++;
                break; // Found a non-zero element in this row, move to the next row
            }
        }
    }
    return rank;
}
\end{lstlisting}
\end{frame}

\begin{frame}[fragile]
\frametitle{Python Code (using C shared library)}
\begin{lstlisting}[language=Python]
import ctypes
import numpy as np

# Load the shared library
lib_code = ctypes.CDLL("./code25.so")

# Define the argument types and return type for the C function
lib_code.calculate_rank.argtypes = [
    ctypes.POINTER(ctypes.c_int), # matrix_ptr (flattened 2D array)
    ctypes.c_int,                  # rows
    ctypes.c_int                   # cols
]
lib_code.calculate_rank.restype = ctypes.c_int
\end{lstlisting}
\end{frame}

\begin{frame}[fragile]
\frametitle{Python Code (using C shared library)}
\begin{lstlisting}[language=Python]
# The matrix from the image
matrix_data = [
    [1, 2, 2, 3],
    [3, 4, 2, 5],
    [5, 6, 2, 7],
    [7, 8, 2, 9]
]

rows = len(matrix_data)
cols = len(matrix_data[0])

# Flatten the matrix into a 1D list for C compatibility
flattened_matrix = [item for sublist in matrix_data for item in sublist]
\end{lstlisting}
\end{frame}

\begin{frame}[fragile]
\frametitle{Python Code (using C shared library)}
\begin{lstlisting}[language=Python]
# Convert the flattened list to a C-compatible array
C_int_array = ctypes.c_int * (rows * cols)
c_matrix = C_int_array(*flattened_matrix)

# Call the C function to calculate the rank
rank = lib_code.calculate_rank(c_matrix, rows, cols)

print(f"The rank of the matrix is: {rank}")
\end{lstlisting}
\end{frame}

\begin{frame}[fragile]
\frametitle{Pure Python Code}
\begin{lstlisting}[language=Python]
import numpy as np  # <-- This line imports the numpy library

# Define the matrix from the image
matrix = np.array([
    [1, 2, 2, 3],
    [3, 4, 2, 5],
    [5, 6, 2, 7],
    [7, 8, 2, 9]
])

# Calculate the rank of the matrix using numpy's linear algebra module
rank = np.linalg.matrix_rank(matrix) 

print(f"The given matrix is:\n{matrix}")
print(f"The rank of the matrix is: {rank}")
\end{lstlisting}
\end{frame}

\end{document}