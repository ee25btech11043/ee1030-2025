\documentclass[journal]{IEEEtran}
\usepackage[a5paper, margin=10mm, onecolumn]{geometry}
\usepackage{lmodern}
\usepackage{tfrupee}
\setlength{\headheight}{1cm}
\setlength{\headsep}{0mm}

\usepackage{gvv-book}
\usepackage{gvv}
\usepackage{cite}
\usepackage{amsmath,amssymb,amsfonts,amsthm}
\usepackage{algorithmic}
\usepackage{graphicx}
\usepackage{textcomp}
\usepackage{xcolor}
\usepackage{txfonts}
\usepackage{listings}
\usepackage{enumitem}
\usepackage{mathtools}
\usepackage{gensymb}
\usepackage{comment}
\usepackage[breaklinks=true]{hyperref}
\usepackage{tkz-euclide}
\usepackage{listings}
\def\inputGnumericTable{}
\usepackage[latin1]{inputenc}
\colorlet{punct}{red!60!black}
\definecolor{background}{HTML}{EEEEEE}
\definecolor{delim}{RGB}{20,105,176}
\colorlet{sqbracket}{delim}
\colorlet{comment}{green!50!black}
\usepackage{array}
\usepackage{longtable}
\usepackage{calc}
\usepackage{multirow}
\usepackage{hhline}
\usepackage{ifthen}
\usepackage{lscape}
\usepackage{xparse}

\bibliographystyle{IEEEtran}

\title{12.160}
\author{EE25BTECH11043 - Nishid Khandagre}
\begin{document}
\maketitle

\renewcommand{\thefigure}{\theenumi}
\renewcommand{\thetable}{\theenumi}

\numberwithin{equation}{enumi}
\numberwithin{figure}{enumi}


\textbf{Question}:\
If $\vec{A}$ is square symmetric real valued matrix of dimension 2n, the eigenvalues of $\vec{A}$ are
\begin{itemize}
    \item a) 2n distinct real values numbers
    \item b) 2n real values, not necessarily distinct
    \item c) n distinct pairs of complex conjugate numbers
    \item d) n pairs of complex conjugate numbers, not necessarily distinct
\end{itemize}

\textbf{Solution: }
Let $\vec{A}$ be a real symmetric matrix of size $2n \times 2n$.
\begin{align}
    \vec{A} &= \vec{A}^T =\bar{A}^{\top}= \vec{A}^*
\end{align}
Let $\vec{v}$ be an eigenvector and $\lambda$ its eigenvalue:
\begin{align}
    \vec{A} \vec{v} &= \lambda \vec{v}
\end{align}
Take Hermitian inner product of both sides with $\vec{v}$:
\begin{align}
    \vec{v}^* \vec{A} \vec{v} &= \lambda \vec{v}^* \vec{v}
\end{align}
Since $\vec{A} = \vec{A}^*$:
\begin{align}
    \vec{v}^* \vec{A} \vec{v} &= (\vec{A} \vec{v})^* \vec{v} \\
    &= (\lambda \vec{v})^* \vec{v} \\
    &= \overline{\lambda} \vec{v}^* \vec{v}
\end{align}
Equating both expressions for $\vec{v}^* \vec{A} \vec{v}$:
\begin{align}
    \lambda \vec{v}^* \vec{v} &= \overline{\lambda} \vec{v}^* \vec{v}
\end{align}
Since $\vec{v}^* \vec{v} > 0$ (as $\vec{v}$ is an eigenvector, it must be non-zero):
\begin{align}
    \lambda &= \overline{\lambda} \\
    \implies \lambda &\text{ is real.}
\end{align}
Therefore, All eigenvalues are real.\\
$\vec{A}$ has 2n eigenvalues since its dimension is $2n \times 2n$.
The eigenvalues may be repeated (not necessarily distinct).\\
Therefore, the correct option is (b).

\end{document}
