\documentclass{beamer}
\usepackage[utf8]{inputenc}
\usetheme{Madrid}
\usecolortheme{default}
\usepackage{amsmath,amssymb,amsfonts,amsthm}
\usepackage{txfonts}
\usepackage{tkz-euclide}
\usepackage{listings}
\usepackage{adjustbox}
\usepackage{array}
\usepackage{tabularx}
\usepackage{gvv}
\usepackage{lmodern}
\usepackage{circuitikz}
\usepackage{tikz}
\usepackage{graphicx}
\setbeamertemplate{page number in head/foot}[totalframenumber]
\usepackage{tcolorbox}
\tcbuselibrary{minted,breakable,xparse,skins}
\definecolor{bg}{gray}{0.95}
\DeclareTCBListing{mintedbox}{O{}m!O{}}{%
breakable=true,
listing engine=minted,
listing only,
minted language=#2,
minted style=default,
minted options={%
linenos,
gobble=0,
breaklines=true,
breakafter=,,
fontsize=\small,
numbersep=8pt,
#1},
boxsep=0pt,
left skip=0pt,
right skip=0pt,
left=25pt,
right=0pt,
top=3pt,
bottom=3pt,
arc=5pt,
leftrule=0pt,
rightrule=0pt,
bottomrule=2pt,
toprule=2pt,
colback=bg,
colframe=orange!70,
enhanced,
overlay={%
\begin{tcbclipinterior}
\fill[orange!20!white] (frame.south west) rectangle ([xshift=20pt]frame.north west);
\end{tcbclipinterior}},
#3,
}
\lstset{
language=C,
basicstyle=\ttfamily\small,
keywordstyle=\color{blue},
stringstyle=\color{orange},
commentstyle=\color{green!60!black},
numbers=left,
numberstyle=\tiny\color{gray},
breaklines=true,
showstringspaces=false,
}

\title
{12.368}
\date{October 8, 2025}
\author
{EE25BTECH11043 - Nishid Khandagre}

\begin{document}

\frame{\titlepage}

\begin{frame}{Question}
If the nullity of the matrix
$\myvec{k & 1 & 2 \\ 1 & -1 & -2 \\ 1 & 1 & 4}$
is 1, then the value of k is ?
\end{frame}

\begin{frame}{Solution}
Given matrix:
\begin{align}
\vec{A}=\myvec{k & 1 & 2 \\ 1 & -1 & -2 \\ 1 & 1 & 4}
\end{align}



Nullity = 1 for a $3 \times 3$ matrix means $\operatorname{rank}(A)=2$.
\end{frame}

\begin{frame}{Solution}
Swap $R_1$ and $R_2$:
\begin{align}
\myvec{1 & -1 & -2 \\ k & 1 & 2 \\ 1 & 1 & 4}
\end{align}

Apply $R_2 \rightarrow R_2 - kR_1$ and $R_3 \rightarrow R_3 - R_1$:
\begin{align}
\myvec{1 & -1 & -2 \\ 0 & 1+k & 2+2k \\ 0 & 2 & 6}
\end{align}
\end{frame}

\begin{frame}{Solution}
For the rank to be 2, the last two rows must be linearly dependent.
Thus, there exists a scalar $t$ such that:
\begin{align}
1+k &= 2t \quad \label{eq:1} \\
2+2k &= 6t \quad \label{eq:2}
\end{align}
Divide the equations
\begin{align}
2+2k &= 3(1+k) \\
2+2k &= 3+3k\\
k &= -1
\end{align}
Thus, the value of $k$ is $-1$.
\end{frame}

\begin{frame}[fragile]
\frametitle{C Code}
\begin{lstlisting}{c}
#include <stdio.h>
// Function to calculate the determinant of a 3x3 matrix
// The matrix is passed as a flat array of 9 doubles for simplicity in ctypes.
//order: [a11, a12, a13, a21, a22, a23, a31, a32, a33]
double calculate_determinant(double* matrix_elements) {
    double a = matrix_elements[0]; double b = matrix_elements[1]; double c = matrix_elements[2];
    double d = matrix_elements[3]; double e = matrix_elements[4]; double f = matrix_elements[5];
    double g = matrix_elements[6]; double h = matrix_elements[7]; double i = matrix_elements[8];

    // Formula for 3x3 determinant:
    // a(ei - fh) - b(di - fg) + c(dh - eg)
    return a * (e * i - f * h) - b * (d * i - f * g) + c * (d * h - e * g);
}
\end{lstlisting}
\end{frame}

\begin{frame}[fragile]
\frametitle{Python Code (using C shared library)}
\begin{lstlisting}{python}
import ctypes
import numpy as np
# Load the shared library
lib_matrix = ctypes.CDLL("./code22.so")
# Define the argument types and return type for the C function
lib_matrix.calculate_determinant.argtypes = [
    ctypes.POINTER(ctypes.c_double)
]
lib_matrix.calculate_determinant.restype = ctypes.c_double
def get_determinant_from_c(k_value):
#Constructs the matrix for a given k_value, flattens it,
#and calls the C function to get its determinant.

    matrix_elements_flat = (ctypes.c_double * 9)(
        k_value, 1.0, 2.0,
        1.0, -1.0, -2.0,
        1.0, 1.0, 4.0
    )
\end{lstlisting}
\end{frame}

\begin{frame}[fragile]
\frametitle{Python Code (using C shared library)}
\begin{lstlisting}{python}
determinant = lib_matrix.calculate_determinant(matrix_elements_flat)
    return determinant
def solve_for_k_with_c_determinant():
#Finds the value of k such that the nullity of the matrix is 1.
#This implies the determinant of the matrix is 0.
    print("Solving for 'k' in the matrix problem where nullity is 1 (determinant = 0).\n")
    k_solution_algebraic = -1.0
    print(f"1. Algebraic Solution: From det = -2k - 2 = 0, we find k = {k_solution_algebraic:.0f}")

    print("\n2. Verification using the C function for k = -1:")
    verified_determinant = get_determinant_from_c(k_solution_algebraic)
    print(f"   For k = {k_solution_algebraic:.0f}, the determinant (from C code) is: {verified_determinant:.6f}")
\end{lstlisting}
\end{frame}

\begin{frame}[fragile]
\frametitle{Python Code (using C shared library)}
\begin{lstlisting}
    if abs(verified_determinant) < 1e-9:
        print("   The determinant is approximately zero, confirming k = -1 is correct.")
    else:
        print("   The determinant is not zero. There might be an issue with calculation or assumption.")

    print(f"\nTherefore, the value of k for which the nullity of the matrix is 1 is: {k_solution_algebraic:.0f}")

    print("\n--- Visualizing the matrix with k = -1 ---")
    final_matrix_k = -1
    final_matrix = np.array([
        [final_matrix_k, 1, 2],
        [1, -1, -2],
        [1, 1, 4]
    ])
    \end{lstlisting}
\end{frame}

\begin{frame}[fragile]
\frametitle{Python Code (using C shared library)}
\begin{lstlisting}
    print(final_matrix)
    print(f"Numpy's determinant for this matrix: {np.linalg.det(final_matrix):.6f}")
    print(f"Numpy's rank for this matrix: {np.linalg.matrix_rank(final_matrix)}")
    print(f"Numpy's nullity (columns - rank): {final_matrix.shape[1] - np.linalg.matrix_rank(final_matrix)}")

# Run the solver
solve_for_k_with_c_determinant()
\end{lstlisting}
\end{frame}

\begin{frame}[fragile]
\frametitle{Python Code (Pure Python)}
\begin{lstlisting}{python}
import numpy as np

def calculate_determinant_pure_python(k_val):
    """
    Calculates the determinant of the given 3x3 matrix using NumPy.
    The matrix structure is:
    [ k  1  2 ]
    [ 1 -1 -2 ]
    [ 1  1  4 ]
    """
    matrix = np.array([
        [k_val, 1, 2],
        [1, -1, -2],
        [1, 1, 4]
    ])
    return np.linalg.det(matrix)
\end{lstlisting}
\end{frame}

\begin{frame}[fragile]
\frametitle{Python Code (Pure Python)}
\begin{lstlisting}{python}
def solve_matrix_problem_pure_python():
    """
    For a 3x3 matrix, nullity 1 means rank is 2.
    For rank to be 2 (and not 3), the determinant must be 0.
    """
    k_solution = -1.0
    print(f"1. Manual Algebraic Derivation:")
    print(f"   The determinant of the matrix is -2k - 2.")
    print(f"   For nullity to be 1, the determinant must be 0.")
    print(f"   Setting -2k - 2 = 0 gives k = {k_solution:.0f}.\n")

    print(f"2. Verification using NumPy's determinant function:")
    matrix_with_k = np.array([
        [k_solution, 1, 2],
        [1, -1, -2],
        [1, 1, 4]
    ])
\end{lstlisting}
\end{frame}

\begin{frame}[fragile]
\frametitle{Python Code (Pure Python)}
\begin{lstlisting}{python}
    print("   Matrix with k = -1:")
    print(matrix_with_k)

    det_verified = np.linalg.det(matrix_with_k)
    print(f"   Determinant (calculated by NumPy): {det_verified:.6f}")

    if abs(det_verified) < 1e-9:
        print("   The determinant is approximately zero, confirming k = -1 is correct for rank < 3.")
    else:
        print("   Error: Determinant is not zero as expected.")

    rank = np.linalg.matrix_rank(matrix_with_k)
    nullity = matrix_with_k.shape[1] - rank
\end{lstlisting}
\end{frame}

\begin{frame}[fragile]
\frametitle{Python Code (Pure Python)}
\begin{lstlisting}{python}
    print(f"   Rank of the matrix: {rank}")
    print(f"   Nullity of the matrix (columns - rank): {nullity}\n")

    if nullity == 1:
        print(f"Conclusion: The value of k that results in a nullity of 1 is: {k_solution:.0f}")
    else:
        print("Conclusion: The calculated k did not result in a nullity of 1.")

# Run the solver
solve_matrix_problem_pure_python()
\end{lstlisting}
\end{frame}

\end{document}