\documentclass{beamer}
\usepackage[utf8]{inputenc}

\usetheme{Madrid}
\usecolortheme{default}
\usepackage{amsmath,amssymb,amsfonts,amsthm}
\usepackage{txfonts}
\usepackage{tkz-euclide}
\usepackage{listings}
\usepackage{adjustbox}
\usepackage{array}
\usepackage{tabularx}
\usepackage{gvv}
\usepackage{lmodern}
\usepackage{circuitikz}
\usepackage{tikz}
\usepackage{graphicx}

\setbeamertemplate{page number in head/foot}[totalframenumber]

\usepackage{tcolorbox}
\tcbuselibrary{minted,breakable,xparse,skins}

\definecolor{bg}{gray}{0.95}
\DeclareTCBListing{mintedbox}{O{}m!O{}}{%
breakable=true,
listing engine=minted,
listing only,
minted language=#2,
minted style=default,
minted options={%
linenos,
gobble=0,
breaklines=true,
breakafter=,,
fontsize=\small,
numbersep=8pt,
#1},
boxsep=0pt,
left skip=0pt,
right skip=0pt,
left=25pt,
right=0pt,
top=3pt,
bottom=3pt,
arc=5pt,
leftrule=0pt,
rightrule=0pt,
bottomrule=2pt,
toprule=2pt,
colback=bg,
colframe=orange!70,
enhanced,
overlay={%
\begin{tcbclipinterior}
\fill[orange!20!white] (frame.south west) rectangle ([xshift=20pt]frame.north west);
\end{tcbclipinterior}},
#3,
}
\lstset{
language=C,
basicstyle=\ttfamily\small,
keywordstyle=\color{blue},
stringstyle=\color{orange},
commentstyle=\color{green!60!black},
numbers=left,
numberstyle=\tiny\color{gray},
breaklines=true,
showstringspaces=false,
}

\title
{12.160}
\date{October 8, 2025}
\author
{EE25BTECH11043 - Nishid Khandagre}

\begin{document}

\frame{\titlepage}

\begin{frame}{Question}
If $\vec{A}$ is square symmetric real valued matrix of dimension 2n, the eigenvalues of $\vec{A}$ are
\begin{itemize}
    \item a) 2n distinct real values numbers
    \item b) 2n real values, not necessarily distinct
    \item c) n distinct pairs of complex conjugate numbers
    \item d) n pairs of complex conjugate numbers, not necessarily distinct
\end{itemize}
\end{frame}

\begin{frame}{Theoretical Solution}
\textbf{Solution: }
Let $\vec{A}$ be a real symmetric matrix of size $2n \times 2n$.
\begin{align}
    \vec{A} &= \vec{A}^T = \vec{A}^*
\end{align}
Let $\vec{v}$ be an eigenvector and $\lambda$ its eigenvalue:
\begin{align}
    \vec{A} \vec{v} &= \lambda \vec{v}
\end{align}
Take Hermitian inner product of both sides with $\vec{v}$:
\begin{align}
    \vec{v}^* \vec{A} \vec{v} &= \lambda \vec{v}^* \vec{v}
\end{align}
\end{frame}

\begin{frame}{Theoretical Solution}
Since $\vec{A} = \vec{A}^*$:
\begin{align}
    \vec{v}^* \vec{A} \vec{v} &= (\vec{A} \vec{v})^* \vec{v} \\
    &= (\lambda \vec{v})^* \vec{v} \\
    &= \overline{\lambda} \vec{v}^* \vec{v}
\end{align}
Equating both expressions for $\vec{v}^* \vec{A} \vec{v}$:
\begin{align}
    \lambda \vec{v}^* \vec{v} &= \overline{\lambda} \vec{v}^* \vec{v}
\end{align}
Since $\vec{v}^* \vec{v} > 0$ (as $\vec{v}$ is an eigenvector, it must be non-zero):
\begin{align}
    \lambda &= \overline{\lambda} \\
    \implies \lambda &\text{ is real.}
\end{align}
\end{frame}

\begin{frame}{Theoretical Solution}
Therefore, All eigenvalues are real.\\\\
$\vec{A}$ has 2n eigenvalues since its dimension is $2n \times 2n$.
The eigenvalues may be repeated (not necessarily distinct).\\\\
Therefore, the correct option is (b).
\end{frame}

\end{document}