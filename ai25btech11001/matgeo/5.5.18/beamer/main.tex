
\documentclass{beamer}
\mode<presentation>
\usepackage{amsmath}
\usepackage{amssymb}
%\usepackage{advdate}
\usepackage{graphicx}
\graphicspath{{../figs/}}
\usepackage{adjustbox}
\usepackage{subcaption}
\usepackage{enumitem}
\usepackage{multicol}
\usepackage{mathtools}
\usepackage{listings}
\usepackage{url}
\def\UrlBreaks{\do\/\do-}
\usetheme{Boadilla}
\usecolortheme{lily}
\setbeamertemplate{footline}
{
  \leavevmode%
  \hbox{%
  \begin{beamercolorbox}[wd=\paperwidth,ht=2.25ex,dp=1ex,right]{author in head/foot}%
    \insertframenumber{} / \inserttotalframenumber\hspace*{2ex} 
  \end{beamercolorbox}}%
  \vskip0pt%
}
\setbeamertemplate{navigation symbols}{}
\let\solution\relax
\usepackage{gvv}
\lstset{
%language=C,
frame=single, 
breaklines=true,
columns=fullflexible
}

\numberwithin{equation}{section}
\title{5.5.18}
\author{AI25BTECH11001 - ABHISEK MOHAPATRA}
% \maketitle
% \newpage
% \bigskip
\begin{document}
{\let\newpage\relax\maketitle}
\renewcommand{\thefigure}{\theenumi}
\renewcommand{\thetable}{\theenumi}



	 	\textbf{Question}:
Find the inverse of the following matrix, using elementary transformations
\begin{align*}
		\myvec{2&3&1\\2&4&1\\3&7&2}
\end{align*}
		

		\textbf{Solution:}
		Given:
		\begin{align}
				\vec{A}\vec{A}^{-1} = \vec{I}
		\end{align}
		\begin{align}
				\myvec{2&3&1\\2&4&1\\3&7&2}\vec{A}^{-1} = \myvec{1&0&0\\0&1&0\\0&0&1}
		\end{align}
		Augumented Matrix:
		\begin{align}
				\augvec{3}{3}{
						2&3&1&1&0&0\\
						2&4&1&0&1&0\\
						3&7&2&0&0&1}
		\end{align}
		\begin{align}
				\xrightarrow[]{R_2\rightarrow R_2-R_1}\augvec{3}{3}{
						2&3&1&1&0&0\\
						0&1&0&-1&1&0\\
						3&7&2&0&0&1}
		\end{align}
		\begin{align}
				\xrightarrow[]{R_3\rightarrow R_3-\frac{3}{2}R_1}\augvec{3}{3}{
						2&3&1&1&0&0\\
						0&1&0&-1&1&0\\
						0&\frac{5}{2}&\frac{1}{2}&-\frac{3}{2}&0&1}
		\end{align}
		\begin{align}
				\xrightarrow[]{R_3\rightarrow R_3-\frac{5}{2}R_2}\augvec{3}{3}{
						2&3&1&1&0&0\\
						0&1&0&-1&1&0\\
						0&0&\frac{1}{2}&1&-\frac{5}{2}&1}
		\end{align}
		\begin{align}
				\xrightarrow[]{R_1\rightarrow R_1-3R_2-2R_3}\augvec{3}{3}{
						2&0&0&2&2&-2\\
						0&1&0&-1&1&0\\
						0&0&\frac{1}{2}&1&-\frac{5}{2}&1}
		\end{align}
		\begin{align}
				\xrightarrow[]{R_1\rightarrow R_1-3R_2-2R_3}\augvec{3}{3}{
						1&0&0&1&1&-1\\
						0&1&0&-1&1&0\\
						0&0&1&2&-5&2}
		\end{align}So,
		\begin{align}
				\vec{A}^{-1} = \myvec{1&1&-1\\-1&1&0\\2&-5&2}
		\end{align}

\end{document}


