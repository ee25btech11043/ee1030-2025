\documentclass{beamer}
\usepackage[utf8]{inputenc}

\usetheme{Madrid}
\usecolortheme{default}
\usepackage{amsmath,amssymb,amsfonts,amsthm}
\usepackage{mathtools}
\usepackage{txfonts}
\usepackage{tkz-euclide}
\usepackage{listings}
\usepackage{adjustbox}
\usepackage{array}
\usepackage{gensymb}
\usepackage{tabularx}
\usepackage{gvv}
\usepackage{lmodern}
\usepackage{circuitikz}
\usepackage{tikz}
\usepackage{multicol}
\lstset{literate={·}{{$\cdot$}}1 {λ}{{$\lambda$}}1 {→}{{$\to$}}1}
\usepackage{graphicx}

\setbeamertemplate{page number in head/foot}[totalframenumber]

\usepackage{tcolorbox}
\tcbuselibrary{minted,breakable,xparse,skins}



\definecolor{bg}{gray}{0.95}
\DeclareTCBListing{mintedbox}{O{}m!O{}}{%
  breakable=true,
  listing engine=minted,
  listing only,
  minted language=#2,
  minted style=default,
  minted options={%
    linenos,
    gobble=0,
    breaklines=true,
    breakafter=,,
    fontsize=\small,
    numbersep=8pt,
    #1},
  boxsep=0pt,
  left skip=0pt,
  right skip=0pt,
  left=25pt,
  right=0pt,
  top=3pt,
  bottom=3pt,
  arc=5pt,
  leftrule=0pt,
  rightrule=0pt,
  bottomrule=2pt,
  toprule=2pt,
  colback=bg,
  colframe=orange!70,
  enhanced,
  overlay={%
    \begin{tcbclipinterior}
    \fill[orange!20!white] (frame.south west) rectangle ([xshift=20pt]frame.north west);
    \end{tcbclipinterior}},
  #3,
}
\lstset{
    language=C,
    basicstyle=\ttfamily\small,
    keywordstyle=\color{blue},
    stringstyle=\color{orange},
    commentstyle=\color{green!60!black},
    numbers=left,
    numberstyle=\tiny\color{gray},
    breaklines=true,
    showstringspaces=false,
}
%------------------------------------------------------------
%This block of code defines the information to appear in the
%Title page
\title %optional
{12.352}
\date{September 17,2025}
%\subtitle{A short story}

\author % (optional)
{Harsha-EE25BTECH11026}



\begin{document}


\frame{\titlepage}


\begin{frame}{Question}
The matrix form of the linear system 
\begin{align*}
    \frac{dx}{dt}=3x-5y
\end{align*}
\begin{align*}
    \frac{dy}{dt}=4x+8y
\end{align*}
is
\begin{enumerate}
\begin{multicols}{2}
    \item $\frac{d}{dt}\myvec{x\\y}=\myvec{3&&-5\\4&&8}\myvec{x\\y}$
    \item $\frac{d}{dt}\myvec{x\\y}=\myvec{3&&8\\4&&-5}\myvec{x\\y}$
    \item $\frac{d}{dt}\myvec{x\\y}=\myvec{4&&-5\\3&&8}\myvec{x\\y}$
    \item $\frac{d}{dt}\myvec{x\\y}=\myvec{4&&8\\3&&-5}\myvec{x\\y}$
\end{multicols}
\end{enumerate}
\end{frame}

\begin{frame}{Theoretical Solution}
The given differential equations,
\begin{align}
    \frac{dx}{dt}=\myvec{3&&-5}\myvec{x\\y} \label{eq:1}
\end{align}
\begin{align}
    \frac{dy}{dt}=\myvec{4&&8}\myvec{x\\y} \label{eq:2}
\end{align}
From ~\eqref{eq:1} and ~\eqref{eq:2},
\begin{align}
    \myvec{\frac{dx}{dt}\\\frac{dy}{dt}}=\myvec{3&&-5\\4&&8}\myvec{x\\y}
\end{align}
\begin{align}
    \implies \frac{d}{dt}\myvec{x\\y}=\myvec{3&&-5\\4&&8}\myvec{x\\y}
\end{align}
\end{frame}

\end{document}