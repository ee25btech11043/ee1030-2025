\documentclass{beamer}
\usepackage[utf8]{inputenc}

\usetheme{Madrid}
\usecolortheme{default}
\usepackage{amsmath,amssymb,amsfonts,amsthm}
\usepackage{mathtools}
\usepackage{txfonts}
\usepackage{tkz-euclide}
\usepackage{listings}
\usepackage{adjustbox}
\usepackage{array}
\usepackage{gensymb}
\usepackage{tabularx}
\usepackage{gvv}
\usepackage{lmodern}
\usepackage{circuitikz}
\usepackage{tikz}
\usepackage{multicol}
\lstset{literate={·}{{$\cdot$}}1 {λ}{{$\lambda$}}1 {→}{{$\to$}}1}
\usepackage{graphicx}

\setbeamertemplate{page number in head/foot}[totalframenumber]

\usepackage{tcolorbox}
\tcbuselibrary{minted,breakable,xparse,skins}



\definecolor{bg}{gray}{0.95}
\DeclareTCBListing{mintedbox}{O{}m!O{}}{%
  breakable=true,
  listing engine=minted,
  listing only,
  minted language=#2,
  minted style=default,
  minted options={%
    linenos,
    gobble=0,
    breaklines=true,
    breakafter=,,
    fontsize=\small,
    numbersep=8pt,
    #1},
  boxsep=0pt,
  left skip=0pt,
  right skip=0pt,
  left=25pt,
  right=0pt,
  top=3pt,
  bottom=3pt,
  arc=5pt,
  leftrule=0pt,
  rightrule=0pt,
  bottomrule=2pt,
  toprule=2pt,
  colback=bg,
  colframe=orange!70,
  enhanced,
  overlay={%
    \begin{tcbclipinterior}
    \fill[orange!20!white] (frame.south west) rectangle ([xshift=20pt]frame.north west);
    \end{tcbclipinterior}},
  #3,
}
\lstset{
    language=C,
    basicstyle=\ttfamily\small,
    keywordstyle=\color{blue},
    stringstyle=\color{orange},
    commentstyle=\color{green!60!black},
    numbers=left,
    numberstyle=\tiny\color{gray},
    breaklines=true,
    showstringspaces=false,
}
%------------------------------------------------------------
%This block of code defines the information to appear in the
%Title page
\title %optional
{12.248}
\date{September 17,2025}
%\subtitle{A short story}

\author % (optional)
{Harsha-EE25BTECH11026}



\begin{document}


\frame{\titlepage}


\begin{frame}{Question}
A square matrix $\vec{A}$ will be lower triangular matrix if and only if $a_{MN}$ represents an element in the $M^{th}$ row and $N^{th}$ column of the matrix
\begin{enumerate}
\begin{multicols}{2}
    \item $a_{MN}=0$,$N>M$
    \item $a_{MN}=0$,$M>N$
    \item $a_{MN} \neq 0$,$M>N$
    \item $a_{MN} \neq 0$,$N>M$
\end{multicols}
\end{enumerate}
\end{frame}

\begin{frame}{Theoretical Solution}
A lower triangular matrix of size $m\times n$ is defined as for any element $a_{ij}$ in the matrix, 
\begin{align}
    a_{ij}=0 \;\forall\;i<j
\end{align}

By the definition, option (1) are correct.
\end{frame}

\begin{frame}[fragile]
    \frametitle{C Code -Checking whether the matrix is lower triangular matrix}

    \begin{lstlisting}[language=C]
#include<stdio.h>

void check_ltm(int m, int n, double matrix[m][n]){
	if(m!=n){
		printf("It is not a lower triangular matrix\n");
		return;
	}
	for(int i=0;i<m;i++){
		for(int j=0;j<n;j++){if(j>i && matrix[i][j]!=0){
			printf("It is not a lower triangular matrix\n");
			return;
			}
		}
	}
	printf("It is a lower triangular matrix\n");
}
    \end{lstlisting}
\end{frame}


\begin{frame}[fragile]
    \frametitle{Python+C code}

    \begin{lstlisting}[language=Python]
import ctypes 
import numpy as np

lib=ctypes.CDLL("./libltm.so")

lib.check_ltm.argtypes=(ctypes.c_int , ctypes.c_int, np.ctypeslib.ndpointer(dtype=np.float64, ndim=2 , flags="C_CONTIGUOUS"))

lib.check_ltm.restype= None

#Example
A=np.matrix([[2,0,0],[3,1,0],[8,7,6]]).astype(np.float64)

lib.check_ltm(A.shape[0],A.shape[1],A)
    \end{lstlisting}
\end{frame}

\begin{frame}[fragile]
    \frametitle{Python code}
    \begin{lstlisting}[language=Python]
import numpy as np

def check_ltm(matrix):
    m,n= np.shape(matrix)
    if(m!=n):
        print("It is not a lower triangular matrix")
        return
    if np.all(matrix[np.triu_indices(m,k=1)]==0):
        print("It is a lower triangular matrix")
    else:
        print("It is not a lower triangular matrix")

def generate_ltm(n, low=0 , high=10 ):
    A= np.random.randint( low , high , size=(n,n))
    return np.tril(A)

L=generate_ltm(4)
print(L)
    \end{lstlisting}   
\end{frame}

\end{document}