\documentclass{beamer}
\usetheme{Madrid}

\usepackage{amsmath, amssymb, amsthm}
\usepackage{graphicx}
\usepackage{gensymb}
\usepackage[utf8]{inputenc}
\usepackage{hyperref}
\usepackage{tikz}


\title{4.12.17 Matgeo}
\author{AI25BTECH11012 - Garige Unnathi}
\date{}

\begin{document}

\frame{\titlepage}

% Question frame
\begin{frame}
\frametitle{Question}
Let $P_1$ : 2x + y - z = 3 and $P_2$ : x + 2y + z = 2 be two planes . Then, which of the following statements is/are TRUE ?
\begin{enumerate}
    \item The line of intersection of P1 and P2 has direction ratios 1,2,-1
    \item The line $\frac{3x-4}{9}$ = $\frac{1-3y}{9} $= $\frac{z}{3}$ is perpendicular to the line of intersection of P1 and P2
    \item The acute angle between $P_1$ and $P_2$ is 60\degree
    \item If $P_3$ is the plane passing through the point (4,2,-2) and perpendicular to the line of intersection of $P_1$ and $P_2$ ,then the distance of the point (2,1,1) from the plane $P_3$ is $\frac{2}{\sqrt{3}}$
\end{enumerate}
\end{frame}


% Solution steps
\begin{frame}
\frametitle{Solution}
Let 
\begin{align}
    P_1 = \begin{bmatrix}2\\1\\-1\end{bmatrix}^{T}\textbf{X} = 3 \\
    P_2 = \begin{bmatrix}1\\2\\1\end{bmatrix}^{T}\textbf{X} = 2\\
    \begin{bmatrix}2 &1 &-1\\1&2&1\end{bmatrix}\textbf{X} = \begin{bmatrix}3\\2\end{bmatrix}
\end{align}
Combining both equations and solving by row reduction we get :
\begin{align}
  \textbf{X} = \begin{bmatrix}0\\ \frac{5}{3} \\ -\frac{4}{3}\end{bmatrix} + \lambda\begin{bmatrix}1\\-1\\1\end{bmatrix}
\end{align}
Hence , the direction ratios of the line of intersection are (1,-1,1) . 
\end{frame}


\begin{frame}
\frametitle{Solution}
So option 1 is false\\
For option 2 :\\
simplifing the line equation we get the line equation to be :
\begin{align}
 \frac{x-\frac{4}{3}}{3} = \frac{y - \frac{1}{3}}{-3} = \frac{z}{3}\\
 \textbf{X} = \begin{bmatrix}-\frac{4}{3}\\- \frac{1}{3}\\0\end{bmatrix} + \mu\begin{bmatrix}3\\-3\\3\end{bmatrix}
\end{align}

solving the equation by row reduction we get  direction ratios of the line to be (3,-3,3) \\
For two lines to be perpendicular :
\begin{align}
   n_1^{T}n_2 = 0
\end{align}

\end{frame}

\begin{frame}
\frametitle{Solution}
For the given lines :

\begin{align}
  \begin{bmatrix}1\\-1\\1\end{bmatrix}^{T} \begin{bmatrix}3 \\-3\\3\end{bmatrix} = 9 
\end{align}
Hence the lines are not perpendicular . So option 2 is also false 

\end{frame}
\begin{frame}
\frametitle{Solution}
We find the angle between two planes by the formula :
\begin{align}
    \cos\theta = \frac{\lvert n_1^{T}n_2 \rvert}{\lVert n_1\rVert\lVert n_2\rVert}
\end{align}
By solving using above equation we get :
\begin{align}
    \cos\theta = \frac{1}{2}
\end{align}
Hence the angle $\theta$ = 60\degree . So option 3 is true 
\end{frame}

\begin{frame}
\frametitle{Solution}
The plane perpendicular to a line has normal or direction ratios equal to the direction ratios of the line that is (1,-1,1)\\
Hence the plane equation can be written as :
\begin{align}
    \begin{bmatrix}1\\-1\\1\end{bmatrix}^T\textbf{X} = c
\end{align}
To find c we can substitute the point (4,2,-2) in the plane equation :
\begin{align}
    \begin{bmatrix}1\\-1\\1\end{bmatrix}^T\begin{bmatrix}4\\2\\-2\end{bmatrix} = 0
\end{align}
Hence the plane equation is :
\begin{align}
    \begin{bmatrix}1\\-1\\1\end{bmatrix}^T\textbf{X} = 0
\end{align}
\end{frame}

\begin{frame}
\frametitle{Solution}
The distance of a point from a plane is given by the equation :
\begin{align}
    \frac{\lvert n^{T}\textbf{P} - c \rvert}{\lVert n \rVert}
\end{align}
Solving using above equation for the point \textbf{P} = $\begin{bmatrix}2\\1\\1\end{bmatrix}$ we get :
 \begin{align}
      \frac{\lvert 2 - 1 +1 \rvert}{\sqrt{3}} = \frac{2}{\sqrt{3}}
 \end{align}

Hence , option $4$ is also true .
Thus options $3$ and $4$ are true 
\end{frame}


\end{document}
