\documentclass{beamer}

\usepackage[utf8]{inputenc}
\usepackage{lmodern} 
\usepackage[utf8]{inputenc}
\usepackage{lmodern} 
\usepackage{multicol}
\usepackage{listings}
\usepackage{xcolor} 
\usepackage{graphicx}
\definecolor{myblue}{RGB}{48, 63, 159}
\setbeamercolor{palette primary}{bg=myblue, fg=white}
\setbeamercolor{structure}{fg=myblue}
\setbeamercolor{frametitle}{bg=myblue, fg=white}
\setbeamercolor{title}{bg=myblue, fg=white}
\setbeamercolor{footlinecolor}{bg=myblue, fg=white}


\defbeamertemplate*{title page}{mytemplate}{
	\vfill
	\begin{center}
		
		\begin{beamercolorbox}[wd=0.8\paperwidth, center, rounded=true, shadow=true]{title}
			\usebeamerfont{title}\inserttitle\par
		\end{beamercolorbox}
		\vspace{2cm} 
		
		\usebeamerfont{author}\insertauthor
		\vspace{1cm} 
		\usebeamerfont{date}\insertdate
	\end{center}
	\vfill
}


\defbeamertemplate*{frametitle}{mytemplate}{
	\begin{beamercolorbox}[wd=\paperwidth, ht=2.5ex, dp=1.5ex, left]{frametitle}
		\hspace{1em}\usebeamerfont{frametitle}\insertframetitle
	\end{beamercolorbox}
}


\setbeamertemplate{footline}{
	\begin{beamercolorbox}[wd=\paperwidth, ht=2.25ex, dp=1ex]{footlinecolor}
		\hspace{1em}\usebeamerfont{author in footline}\insertshortauthor
		\hfill
		\usebeamerfont{title in footline}\insertshorttitle
		\hfill
		\usebeamerfont{date in footline}\insertdate \hspace{1em} \insertframenumber/\inserttotalframenumber \hspace{0.5em}
	\end{beamercolorbox}
}


\setbeamerfont{author in footline}{size=\tiny}
\setbeamerfont{title in footline}{size=\tiny}
\setbeamerfont{date in footline}{size=\tiny}

\newcommand{\myvec}[1]{\ensuremath{\begin{pmatrix}#1\end{pmatrix}}}
\providecommand{\brak}[1]{\ensuremath{\left(#1\right)}}







\begin{document}
\title{5.5.5}
\author{Shriyansh Chawda-EE25BTECH11052}
\setbeamertemplate{footline}{} 
\frame{\titlepage}

\begin{frame}{Question} 
If $A = \myvec{1 & -1 & 0 \\ 2 & 3 & 4 \\ 0 & 1 & 2}$ and $B = \myvec{2 & 2 & -4 \\ -4 & 2 & -4 \\ 2 & -1 & 5}$, then which of the following is true?
	\begin{enumerate}[a.]
	\begin{multicols}{4}
		\item $A^{-1} = B$
		\item $A^{-1} = 6B$
		\item $B^{-1} = B$
		\item $\mathbf{B^{-1} = \tfrac{1}{6}A}$
	\end{multicols}
\end{enumerate}
\hfill{(12, 2021)}\\
\end{frame}
	
\begin{frame}{Solution}
Given the matrices:
\begin{align}
	A = \myvec{ 1 & -1 & 0 \\ 2 & 3 & 4 \\ 0 & 1 & 2 } \quad B = \myvec{ 2 & 2 & -4 \\ -4 & 2 & -4 \\ 2 & -1 & 5 }
\end{align}
We will find the inverse of A using Gauss-Jordan elimination method.\\
The augmented matrix $[A | I]$ is given by:
\begin{align}
	[A | I] &= \left[ \begin{array}{ccc|ccc} 1 & -1 & 0 & 1 & 0 & 0 \\ 2 & 3 & 4 & 0 & 1 & 0 \\ 0 & 1 & 2 & 0 & 0 & 1 \end{array} \right]
\end{align}
Performing elementary Row Operations\\
$R_2 \to R_2 - 2R_1$:
\begin{align}
	&\left[ \begin{array}{ccc|ccc} 1 & -1 & 0 & 1 & 0 & 0 \\ 0 & 5 & 4 & -2 & 1 & 0 \\ 0 & 1 & 2 & 0 & 0 & 1 \end{array} \right]
\end{align}
\end{frame}

\begin{frame}{Solution}
	Swap rows $R_2 \leftrightarrow R_3$
\begin{align}
	&\left[ \begin{array}{ccc|ccc} 1 & -1 & 0 & 1 & 0 & 0 \\ 0 & 1 & 2 & 0 & 0 & 1 \\ 0 & 5 & 4 & -2 & 1 & 0 \end{array} \right]
\end{align}
$R_1 \to R_1 + R_2$ and $R_3 \to R_3 - 5R_2$
\begin{align}
	&\left[ \begin{array}{ccc|ccc} 1 & 0 & 2 & 1 & 0 & 1 \\ 0 & 1 & 2 & 0 & 0 & 1 \\ 0 & 0 & -6 & -2 & 1 & -5 \end{array} \right]
\end{align}
$R_3 \to -\frac{1}{6}R_3$:
\begin{align}
	&\left[ \begin{array}{ccc|ccc} 1 & 0 & 2 & 1 & 0 & 1 \\ 0 & 1 & 2 & 0 & 0 & 1 \\ 0 & 0 & 1 & \frac{1}{3} & -\frac{1}{6} & \frac{5}{6} \end{array} \right]
\end{align}
\end{frame}
\begin{frame}{Solution}
$R_1 \to R_1 - 2R_3$ and $R_2 \to R_2 - 2R_3$:
\begin{align}
	[I | A^{-1}] &= \left[ \begin{array}{ccc|ccc} 1 & 0 & 0 & \frac{1}{3} & \frac{1}{3} & -\frac{2}{3} \\ 0 & 1 & 0 & -\frac{2}{3} & \frac{1}{3} & -\frac{2}{3} \\ 0 & 0 & 1 & \frac{1}{3} & -\frac{1}{6} & \frac{5}{6} \end{array} \right]
\end{align}
Hence,the Inverse Matrix {$A^{-1}$} is given by 
\begin{align}
	A^{-1} &= \myvec{ \frac{1}{3} & \frac{1}{3} & -\frac{2}{3} \\[1ex] -\frac{2}{3} & \frac{1}{3} & -\frac{2}{3} \\[1ex] \frac{1}{3} & -\frac{1}{6} & \frac{5}{6} }
\end{align}
By factoring out a scalar,The relation with B is given by:
\begin{align}
	A^{-1} = \frac{1}{6} \myvec{ 2 & 2 & -4 \\ -4 & 2 & -4 \\ 2 & -1 & 5}
	\implies A^{-1} = \frac{1}{6}B 
\end{align}
Now Pre-multiplying both side by A:


\end{frame}
\begin{frame}{Solution}
\begin{align}
	A A^{-1} &= A \brak{\frac{1}{6}B} \\
	I &= \frac{1}{6}AB \\
	6I &= AB 
\end{align}
Now Post-multiplying both sides by $B^{-1}$:
\begin{align}
	6I B^{-1} &= A B B^{-1} \\
	6B^{-1} &= A \\
	\mathbf{B^{-1}} &= \mathbf{\frac{1}{6}A}
\end{align}
\end{frame}
\end{document}