
\documentclass[article]{IEEEtran}
\usepackage[a5paper, margin=10mm, onecolumn]{geometry}

\usepackage{tfrupee} 
\setlength{\headheight}{1cm} 
\setlength{\headsep}{0mm}       
\usepackage{multicol}
\usepackage{gvv-book}
\usepackage{gvv}
\usepackage{cite}
\usepackage{amsmath,amssymb,amsfonts,amsthm}
\usepackage{algorithmic}
\usepackage{graphicx}
\usepackage{textcomp}
\usepackage{xcolor}
\usepackage{txfonts}
\usepackage{listings}
\usepackage{enumitem}
\usepackage{mathtools}
\usepackage{gensymb}
\usepackage{comment}
\usepackage[breaklinks=true]{hyperref}
\usepackage{tkz-euclide} 
\usepackage{listings}

\def\inputGnumericTable{}                                 
\usepackage[latin1]{inputenc}                                
\usepackage{color}                                            
\usepackage{array}                                            
\usepackage{longtable}                                       
\usepackage{calc}                                             
\usepackage{multirow}                                         
\usepackage{hhline}                                           
\usepackage{ifthen}                                           
\usepackage{lscape}
\begin{document}
	\title{5.5.5}
	\author{EE25BTECH11052 - Shriyansh Kalpesh Chawda}
	\maketitle
\textbf{Question}\\
	If $A = \myvec{1 & -1 & 0 \\ 2 & 3 & 4 \\ 0 & 1 & 2}$ and $B = \myvec{2 & 2 & -4 \\ -4 & 2 & -4 \\ 2 & -1 & 5}$, then which of the following is true?\\
	\begin{enumerate}
		\begin{multicols}{4}
			\item $A^{-1} = B$
			\item $A^{-1} = 6B$
			\item $B^{-1} = B$
			\item $\mathbf{B^{-1} = \tfrac{1}{6}A}$
		\end{multicols}
	\end{enumerate}
	\hfill{(12, 2021)}\\
	\textbf{Solution}\\
	Given the matrices:
	\begin{align}
		A = \myvec{ 1 & -1 & 0 \\ 2 & 3 & 4 \\ 0 & 1 & 2 } \quad B = \myvec{ 2 & 2 & -4 \\ -4 & 2 & -4 \\ 2 & -1 & 5 }
	\end{align}
	We will find the inverse of A using Gauss-Jordan elimination method.\\
	The augmented matrix $[A | I]$ is given by:
	\begin{align}
		[A | I] &= \left[ \begin{array}{ccc|ccc} 1 & -1 & 0 & 1 & 0 & 0 \\ 2 & 3 & 4 & 0 & 1 & 0 \\ 0 & 1 & 2 & 0 & 0 & 1 \end{array} \right]
	\end{align}
Performing elementary Row Operations\\
 $R_2 \to R_2 - 2R_1$:
	\begin{align}
		&\left[ \begin{array}{ccc|ccc} 1 & -1 & 0 & 1 & 0 & 0 \\ 0 & 5 & 4 & -2 & 1 & 0 \\ 0 & 1 & 2 & 0 & 0 & 1 \end{array} \right]
	\end{align}
	Swap rows $R_2 \leftrightarrow R_3$
	\begin{align}
		&\left[ \begin{array}{ccc|ccc} 1 & -1 & 0 & 1 & 0 & 0 \\ 0 & 1 & 2 & 0 & 0 & 1 \\ 0 & 5 & 4 & -2 & 1 & 0 \end{array} \right]
	\end{align}
		$R_1 \to R_1 + R_2$ and $R_3 \to R_3 - 5R_2$
	\begin{align}
&\left[ \begin{array}{ccc|ccc} 1 & 0 & 2 & 1 & 0 & 1 \\ 0 & 1 & 2 & 0 & 0 & 1 \\ 0 & 0 & -6 & -2 & 1 & -5 \end{array} \right]
	\end{align}
 $R_3 \to -\frac{1}{6}R_3$:
	\begin{align}
		&\left[ \begin{array}{ccc|ccc} 1 & 0 & 2 & 1 & 0 & 1 \\ 0 & 1 & 2 & 0 & 0 & 1 \\ 0 & 0 & 1 & \frac{1}{3} & -\frac{1}{6} & \frac{5}{6} \end{array} \right]
	\end{align}
$R_1 \to R_1 - 2R_3$ and $R_2 \to R_2 - 2R_3$:
	\begin{align}
		[I | A^{-1}] &= \left[ \begin{array}{ccc|ccc} 1 & 0 & 0 & \frac{1}{3} & \frac{1}{3} & -\frac{2}{3} \\ 0 & 1 & 0 & -\frac{2}{3} & \frac{1}{3} & -\frac{2}{3} \\ 0 & 0 & 1 & \frac{1}{3} & -\frac{1}{6} & \frac{5}{6} \end{array} \right]
	\end{align}
Hence,the Inverse Matrix {$A^{-1}$} is given by 
	\begin{align}
		A^{-1} &= \myvec{ \frac{1}{3} & \frac{1}{3} & -\frac{2}{3} \\[1ex] -\frac{2}{3} & \frac{1}{3} & -\frac{2}{3} \\[1ex] \frac{1}{3} & -\frac{1}{6} & \frac{5}{6} }
	\end{align}
	By factoring out a scalar,The relation with B is given by:
	\begin{align}
		A^{-1} = \frac{1}{6} \myvec{ 2 & 2 & -4 \\ -4 & 2 & -4 \\ 2 & -1 & 5}
		\implies A^{-1} = \frac{1}{6}B 
			\end{align}
Now Pre-multiplying both side by A:
	\begin{align}
		A A^{-1} &= A \brak{\frac{1}{6}B} \\
		I &= \frac{1}{6}AB \\
		6I &= AB 
	\end{align}
Now Post-multiplying both sides by $B^{-1}$:
	\begin{align}
		6I B^{-1} &= A B B^{-1} \\
		6B^{-1} &= A \\
		\mathbf{B^{-1}} &= \mathbf{\frac{1}{6}A}
			\end{align}
	
\end{document}