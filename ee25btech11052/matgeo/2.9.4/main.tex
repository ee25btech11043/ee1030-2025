\let\negmedspace\undefined
\let\negthickspace\undefined
\documentclass[journal]{IEEEtran}
\usepackage[a5paper, margin=10mm, onecolumn]{geometry}

\usepackage{tfrupee} 

\setlength{\headheight}{1cm} 
\setlength{\headsep}{0mm}      

\usepackage{gvv-book}
\usepackage{gvv}
\usepackage{cite}
\usepackage{amsmath,amssymb,amsfonts,amsthm}
\usepackage{algorithmic}
\usepackage{graphicx}
\usepackage{textcomp}
\usepackage{xcolor}
\usepackage{txfonts}
\usepackage{listings}
\usepackage{enumitem}
\usepackage{mathtools}
\usepackage{gensymb}
\usepackage{comment}
\usepackage[breaklinks=true]{hyperref}
\usepackage{tkz-euclide} 
\usepackage{listings}

\def\inputGnumericTable{}                                 
\usepackage[latin1]{inputenc}                                
\usepackage{color}                                            
\usepackage{array}                                            
\usepackage{longtable}                                       
\usepackage{calc}                                             
\usepackage{multirow}                                         
\usepackage{hhline}                                           
\usepackage{ifthen}                                           
\usepackage{lscape}





\renewcommand{\thefigure}{\theenumi}
\renewcommand{\thetable}{\theenumi}
\setlength{\intextsep}{10pt} 

\numberwithin{equation}{enumi}
\numberwithin{figure}{enumi}
\renewcommand{\thetable}{\theenumi}
\begin{document}
	
	\bibliographystyle{IEEEtran}
	\vspace{3cm}
	
	\title{2.9.4}
	\author{EE25BTECH11052 - Shriyansh Kalpesh Chawda}
\maketitle
	\textbf{Question}:\\
	\\
	If  $\vec{a} = \hat{i} + \hat{j} + \hat{k}, \quad \vec{a} \cdot \vec{b} = 1, \quad \text{and} \quad \vec{a} \times \vec{b} = \hat{j} - \hat{k},$ then find $|\vec{b}|$.
	\hfill (12, 2022)
	\\
	\solution\\
Given in the question :
\begin{align}
\vec{a} = \myvec{1 \\ 1 \\ 1}.\\
\vec{a} \times \vec{b} = \myvec{0 \\ 1 \\ -1}.\\
\vec{b} = \myvec{b_1 \\ b_2 \\ b_3}
\end{align}
From the dot product: \\
\begin{align}
\vec{a}^\top \vec{b} = 1 \implies
\begin{pmatrix} 1 & 1 & 1 \end{pmatrix} \myvec{b_1 \\ b_2 \\ b_3} = 1\\
b_1 + b_2 + b_3 = 1 
\end{align}
\subsection*{Applying the Cross Product Formula given in the book (2.1.9):}
The provided formula for the cross product is:
\[
A \times B = \myvec{|A_{23} B_{23}| \\ |A_{31} B_{31}| \\ |A_{12} B_{12}|}
\]
where $|A_{ij} B_{ij}|$ represents the determinant of the 2x2 matrix formed by the column vectors $A_{ij}$ and $B_{ij}$.

\subsubsection*{Step 3.1: Define the sub-matrices $A_{ij}$ and $B_{ij}$}
Based on the definition $A_{ij} = \myvec{a_i \\ a_j}$, we define the following matrices for A:
\begin{align}
A_{23} = \myvec{a_2 \\ a_3} = \myvec{1 \\ 1}\\
A_{31} = \myvec{a_3 \\ a_1} = \myvec{1 \\ 1}\\
A_{12} = \myvec{a_1 \\ a_2} = \myvec{1 \\ 1}
\end{align}
Similarly, for the unknown vector B:
\begin{align}
B_{23} = \myvec{b_2 \\ b_3}\\
B_{31} = \myvec{b_3 \\ b_1}\\
B_{12} = \myvec{b_1 \\ b_2}
\end{align}

\subsubsection*{Step 3.2: Calculate the determinants}
Now, we compute the determinants for each component of the cross product:
\begin{align}
	|\vec{A}_{23}\vec{B}_{23}| 
	&= \vec{A}_{23}^\top \vec{B}_{23} 
	= \myvec{1 & b_2 \\ 1 & b_3} 
	= (1)(b_3) - (1)(b_2) 
	= b_3 - b_2 \\	
	|\vec{A}_{31}\vec{B}_{31}| 
	&= \vec{A}_{31}^\top \vec{B}_{31} 
	= \myvec{1 & b_3 \\ 1 & b_1} 
	= (1)(b_1) - (1)(b_3) 
	= b_1 - b_3 \\
	|\vec{A}_{12}\vec{B}_{12}| 
	&= \vec{A}_{12}^\top \vec{B}_{12} 
	= \myvec{1 & b_1 \\ 1 & b_2} 
	= (1)(b_2) - (1)(b_1) 
	= b_2 - b_1 
\end{align}

\begin{align}
\vec{a} \times \vec{b} = \myvec{b_3 - b_2 \\ b_1 - b_3 \\ b_2 - b_1}.
\end{align}


\subsection*{4. Formulating and Solving the System of Equations}
By equating our calculated cross product with the given one, $\mathbf{a} \times \mathbf{b} = \brak{0, 1, -1}$, we get a system of linear equations:
\begin{enumerate}
	\item $b_3 - b_2 = 0$
	\item $b_1 - b_3 = 1$
	\item $b_2 - b_1 = -1$
\end{enumerate}
We also use the given dot product information: $\mathbf{a} \cdot \mathbf{b} = 1$.
\begin{align*}
	a_1b_1 + a_2b_2 + a_3b_3 &= 1 \\
	\brak{1}b_1 + \brak{1}b_2 + \brak{1}b_3 &= 1 \\
\end{align*}
\begin{enumerate}
	\setcounter{enumi}{3}
	\item $b_1 + b_2 + b_3 = 1$
\end{enumerate}
Now we solve this system of four equations:
\begin{itemize}
	\item From equation (1), we find:
	\[ b_3 = b_2 \]
	\item Substitute $b_3 = b_2$ into equation (2):
	\[ b_1 - b_2 = 1 \]
	(Note: This is consistent with equation (3), as multiplying by -1 gives $b_2 - b_1 = -1$)
	\item Now we have a simplified system:
	\begin{itemize}
		\item[(i)] $b_3 = b_2$
		\item[(ii)] $b_1 = 1 + b_2$
		\item[(iii)] $b_1 + b_2 + b_3 = 1$
	\end{itemize}
	\item Substitute (i) and (ii) into (iii):
	\begin{align*}
		\brak{1 + b_2} + b_2 + \brak{b_2} &= 1 \\
		1 + 3b_2 &= 1 \\
		3b_2 &= 0 \\
		b_2 &= 0
	\end{align*}
	\item Now find $b_1$ and $b_3$:
	\begin{align*}
		b_3 &= b_2 = 0 \\
		b_1 &= 1 + b_2 = 1 + 0 = 1
	\end{align*}
\end{itemize}
So, the components of vector $\mathbf{b}$ are $\brak{b_1, b_2, b_3} = \brak{1, 0, 0}$. This means the vector is $\mathbf{b} = 1\hat{i} + 0\hat{j} + 0\hat{k} = \hat{i}$.

To find magnitude,
\begin{align}
	\vec{b}^\top \vec{b} = 1 \\
	\myvec{ 1 & 0 & 0 } \myvec{1 \\ 0 \\ 0} = 1
\end{align}
The magnitude of vector $\vec{b}$ is \textbf{1}.

		

\end{document}